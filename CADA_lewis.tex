\documentclass[man, noapacite, 12pt]{apa2}

\makeatletter
\newenvironment{chapquote}[2][2em]
  {\setlength{\@tempdima}{#1}%
   \def\chapquote@author{#2}%
   \parshape 1 \@tempdima \dimexpr\textwidth-2\@tempdima\relax%
   \itshape}
  {\par\normalfont\hfill--\ \chapquote@author\hspace*{\@tempdima}\par\bigskip}
\makeatother

%\usepackage{pdfsync}
\usepackage{amsmath}
%\usepackage{graphicx}
%\usepackage{topcapt}
%\usepackage{color}
%\usepackage{comment}
\usepackage{booktabs}
\usepackage{apacite2}
\usepackage{fullpage,rotating}
\usepackage{pslatex}
\usepackage{amssymb}


\title{Linguistic structure emerges from cognitive mechanisms}
\author{Molly L. Lewis}
\affiliation{Department of Psychology, Stanford University\\ Conceptual Analysis of Dissertation Area\\ 6 October 2014}


\shorttitle{Linguistic structure emerges from cognitive mechanisms}
\rightheader{Linguistic structure emerges from cognitive mechanisms}
\acknowledgements{Advisor: Michael C. Frank\\ \noindent Additional Readers: Ellen Markman and Noah Goodman}

\abstract{People interact with people and, because of shared interest, try to coordinate their behavior. This behavior is governed by the competition of pragmatic pressures. These pressures lead to equilibria. These equilibria become conventionalized over time (Lewis). This convention becomes an additional pressure in the moment of interaction. Language as a paradigm case of these dynamics.}

\begin{document}
\maketitle

%%%%%%%%% INTRO %%%%%%%%% 
%\begin{chapquote}{Aristotle, \textit{Nic. Ethics II 6}}
%\noindent  Virtue, then, is a state of character concerned with choice, lying in a mean, i.e. the mean relative to us, this being determined by a rational principle, and by that principle by which the man of practical wisdom would determine it. Now it is a mean between two vices, that which depends on excess and that which depends on defect; and again it is a mean because the vices respectively fall short of or exceed what is right in both passions and actions, while virtue both finds and chooses that which is intermediate. Hence in respect of its substance and the definition which states its essence virtue is a mean, with regard to what is best and right an extreme.
%\end{chapquote}

\begin{chapquote}{G.K. Zipf, \textit{1949}}
\noindent  Human society can be view as a field which both influences the individual members of the group and is influenced by them. 
\end{chapquote}

\section{Introduction}
``Room for cream?'' asked the barista. ``Mm, yes -- just a bit" replied the customer. Mundane linguistic interactions such as this are the building blocks of daily experience. They are individuals making sounds to each other in an effort to coordinate their behavior in the physical world \cite{clark2006social}. These interactions are messy, variable, and highly unconstrained. Indeed it is this variability that gives language its vast expressive power \cite{hockett1960}. Yet, despite this appearance of irregularity, rich patterns in linguistic usage are revealed when we aggregate across instances of language use both within and across languages. At the level of syntax, for example, there is a strong bias in English to put subjects before verbs and, across languages, this pattern is attested more often than would be expected by chance alone  \cite{dryer2005order}. These types of probabilistic regularities exist at every level of linguistic structure --- from phonology, to semantics, syntax, and discourse --- and researchers from a variety of disciplines have taken as their project the goal of characterizing these regularities.

In this paper, I argue that we can gain  insight into the character of linguistic structure by considering the dynamics of language use. I will suggest the best way to do this is by framing language use as an instance of a broader phenomenon: social interaction \cite{clark1996using}. In particular, I will adopt the formal framework of social interaction proposed by Thomas Schelling, in which social interactions are viewed as acts of solving coordination problems.  To illustrate, consider the barista example above. In this example, the agents are the barista and the customer, and they must coordinate how  to fill the coffee mug. There are two outcomes --- full and almost full --- and the barista's desired outcome is determined by the preference of the customer. In this case, the barista and the customer rely on language to coordinate their behavior, but this coordination could have been achieved in other ways (e.g. the customer could have shook her head, pointed to the place inside the mug that she wanted the coffee filled to, etc.). Coordination of their behavior is achieved  by arriving at the mutually preferred outcome (the customer's mug is almost full).

A key tenet to the broader argument is that the act of using language is itself an act of solving a coordination problem \cite{clark1996using} . When a person speakers, there are many possible ways the utterance could be interpreted, and arriving at the intended interpretation is an act of coordination. In the case of the barista's, ``Room for cream?," for example. Following \citeA{lewis1969convention}, I will suggest that we can gain understanding of the linguistic coordination problem by using Schelling's formal framework.  This perspective on language use will provide a powerful framework for understanding the relationship between language use and language structure.
%cite clark 1996

In Part I, I will outline the linguistic coordination problem as a paradigmatic case of the social coordination problem. I will suggest that coordination problems are solved through the dynamics of two opposing two forces --- the goals of the one's self and the goal of the other. Following Lewis, I suggest that these opposing forces are resolved by finding an equilibrium point. I will then argue that the equilibria that are reached in language use are reflected in the structure of language, and survey a variety of phenomena in linguistic structure that show this pattern.

In Part II, I will consider the mechanism that might cause linguistic structure to reflect the equilibria reached in linguistic use. Lewis argued that once individuals succeed in solving a coordination problem, there is a tendency to stick with that solution (even though other, equally good solutions exist). This solution is called a convention. I will argue that the key to understanding the link between regularities in linguistic usage and linguistic structure is the process of children's acquisition of these conventions. I will consider how the dynamics of language acquisition might lead to language change. Finally, I will consider the complementary role of conventions and cognitive forces in solving language coordination problems, and the  empirical challenge associated with disentangling the two.

%%%%%%%%% PART I %%%%%%%%% 
\section{Part I: Linguistic structure reflects pragmatic equilibria}
\subsection{Social interaction as coordination}

.focus proposed by David \citeA{lewis1969convention}, the proposal is that all social interactions, including language use, are coordination problems. Lewis defines a coordination problem as follows: ``Two or more agents must each choose one of several alternative actions. Often all the agents have the same set of alternative actions, but that is not necessary. The outcomes the agents want to produce or prevent are determined jointly by the actions of all the agents."

\subsection{The pragmatic forces of coordination}

\subsection{Pragmatic equilibria in the structure of language}

%%%%%%%%% PART II %%%%%%%%% 
\section{Part II: Ontogenesis as the mechanism for the emergence of linguistic structure from pragmatic equilibria}


\bibliographystyle{apacite}
\bibliography{biblibrary}

\end{document}

