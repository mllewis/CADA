\documentclass[man, noapacite, 12pt]{apa2}

\makeatletter
\newenvironment{chapquote}[2][2em]
  {\setlength{\@tempdima}{#1}%
   \def\chapquote@author{#2}%
   \parshape 1 \@tempdima \dimexpr\textwidth-2\@tempdima\relax%
   \itshape}
  {\par\normalfont\hfill--\ \chapquote@author\hspace*{\@tempdima}\par\bigskip}
\makeatother

%\usepackage{pdfsync}
\usepackage{amsmath}
\usepackage{graphicx}
%\usepackage{topcapt}
%\usepackage{color}
%\usepackage{comment}
\usepackage{booktabs}
\usepackage{apacite2}
\usepackage{fullpage,rotating}
\usepackage{pslatex}
\usepackage{amssymb}
\usepackage{multirow}


\title{Linguistic structure emerges from cognitive mechanisms}
\author{Molly L. Lewis}
\affiliation{Department of Psychology, Stanford University\\ Conceptual Analysis of Dissertation Area\\ 6 October 2014}


\shorttitle{Linguistic structure emerges from cognitive mechanisms}
\rightheader{Linguistic structure emerges from cognitive mechanisms}
\acknowledgements{Advisor: Michael C. Frank\\ \noindent Additional Readers: Ellen Markman and Noah Goodman}

\abstract{People interact with people and, because of shared interest, try to coordinate their behavior. This behavior is governed by the competition of pragmatic pressures. These pressures lead to equilibria. These equilibria become conventionalized over time (Lewis). This convention becomes an additional pressure in the moment of interaction. Language as a paradigm case of these dynamics.}

\begin{document}
\maketitle

%%%%%%%%% INTRO %%%%%%%%% 
%\begin{chapquote}{Aristotle, \textit{Nic. Ethics II 6}}
%\noindent  Virtue, then, is a state of character concerned with choice, lying in a mean, i.e. the mean relative to us, this being determined by a rational principle, and by that principle by which the man of practical wisdom would determine it. Now it is a mean between two vices, that which depends on excess and that which depends on defect; and again it is a mean because the vices respectively fall short of or exceed what is right in both passions and actions, while virtue both finds and chooses that which is intermediate. Hence in respect of its substance and the definition which states its essence virtue is a mean, with regard to what is best and right an extreme.
%\end{chapquote}

\begin{chapquote}{G.K. Zipf, \textit{1949}}
\noindent  Human society can be viewed as a field which both influences the individual members of the group and is influenced by them. 
\end{chapquote}

\section{Introduction}
``Room for cream?'' asked the barista. ``Mm, yes -- just a bit" replied the customer. Mundane linguistic interactions such as this are the building blocks of daily experience. They are individuals making sounds to each other in an effort to coordinate their behavior in the physical world \cite{clark2006social}. These interactions are messy, variable, and highly unconstrained. Indeed it is this variability that gives language its vast expressive power \cite{hockett1960}. Yet, despite this appearance of irregularity, rich patterns in linguistic usage are revealed when we aggregate across instances of language use both within and across languages. At the level of syntax, for example, there is a strong bias in English to put subjects before verbs and, across languages, this pattern is attested more often than would be expected by chance alone  \cite{dryer2005order}. These types of probabilistic regularities exist at every level of linguistic structure --- from phonology, to semantics, syntax, and discourse --- and researchers from a variety of disciplines have taken as their project the goal of characterizing these regularities.

In this paper, I argue that we can gain  insight into the character of linguistic structure by considering the dynamics of language use. I will suggest the best way to do this is by framing language use as an instance of a broader phenomenon: social interaction \cite{clark1996using}. In particular, I will adopt the formal framework of social interaction proposed by Thomas Schelling, in which social interactions are viewed as acts of solving coordination problems.  To illustrate, consider the barista example above. In this example, the agents are the barista and the customer, and they must coordinate how  to fill the coffee mug. There are two outcomes --- full and almost full --- and the barista's desired outcome is determined by the preference of the customer. In this case, the barista and the customer rely on language to coordinate their behavior, but this coordination could have been achieved in other ways (e.g. the customer could have shook her head, pointed to the place inside the mug that she wanted the coffee filled to, etc.). Coordination of their behavior is achieved  by arriving at the mutually preferred outcome (the customer's mug is almost full).

A key tenet to the broader argument is that the act of using language is itself an act of solving a coordination problem \cite{clark1996using}. When a person speaks, there are many possible ways the utterance could be interpreted, and arriving at the intended interpretation is an act of coordination. For example, in the case of the customer's interaction with the barista, there are many possible interpretations of the phrase, ``Room for cream?." The barista could mean ``Would you like to add cream to your coffee? If so, I will facilitate that by not filling your mug full with coffee." Or, ``We have so much extra inventory of cream! Do you have room in your bag to take some?" Or, ``Do you like the band `Room for cream'?". Or, if the speaker is speaking another language, a totally unrelated meaning. The point is that the speaker's intended meaning is underspecified from the language alone and the interlocutors must work collaboratively to arrive at a shared understanding. Following \citeA{lewis1969convention}, I will suggest that we can gain insight into the dynamics of linguistic coordination problems such as this by using Schelling's formal framework.  This perspective on language use will ultimately provide a helpful framework for understanding the relationship between language use and language structure.

In Part I, I will outline the linguistic coordination problem as a paradigmatic case of the social coordination problem. I will suggest that coordination problems are solved through the dynamics of two opposing two forces --- the goals of the one's self and the goal of the other. Following Lewis, I suggest that these opposing forces are resolved by finding an equilibrium point. I will then argue that the equilibria that are reached in language use are reflected in the structure of language, and survey a variety of phenomena in linguistic structure that show this pattern.

In Part II, I will consider the mechanism that might cause linguistic structure to reflect the equilibria reached in linguistic use. Lewis argued that once individuals succeed in solving a coordination problem, there is a tendency to stick with that solution (even though other, equally good solutions exist). This solution is called a convention. I will argue that the key to understanding the link between regularities in linguistic usage and linguistic structure is the process of children's acquisition of these conventions. I will consider how the dynamics of language acquisition might lead to language change. Finally, I will consider the complementary role of conventions and cognitive forces in solving language coordination problems, and the  empirical challenge associated with disentangling the two.

%%%%%%%%% PART I %%%%%%%%% 
\section{Part I: Linguistic structure reflects pragmatic equilibria}

Where does linguistic structure come from? \citeA[2010]{christiansen2008} argue that  multiple cognitive constraints influence language evolution. They suggest four constraints: the representational format of thought, properties of the percepto-motor system, learning and processing constraints, and so-called `pragmatic' constraints. Pragmatic constraints are the result of reasoning about another speaker's intention in context. Their argument is that these  cognitive constraints  influence  language at the moment of use, but over time, these biases become instantiated in the structure of language. Although each of these constraints likely plays an important role in the evolution of language, the present paper focuses on the independent contribution of pragmatic constraints. The claim is that in-the-moment pragmatic constraints become fossilized in the structure of language over time. To develop this claim, we begin by modeling language use as a type of social coordination.

\subsection{Social interaction as a coordination problem}
The idea that language is an instance of a much broader class of behavioral phenomenon --- social coordination --- has been noted by many theorist of language \cite{zipf1936, lewis1969convention, grice1975logic, clark1996using}. Each viewed language use as a case of multiple agents making interdependent  rational choices.\footnote{In outlining his seminal theory of pragmatics, Grice  writes: ``As one of my avowed aims is to see talking as a special case or variety of purposive, indeed rational, behaviour, it may be worth noting that the specific expectations of presumptions connected with at least some of the foregoing maxims have their analogues in the sphere of transactions that are not talk exchanges" \cite[pg. 47]{grice1975logic}.} Nonetheless, language use is a paradigmatic case of a coordination problem: it provides a tool that is universal in a community, easy to use, and capable of expressing complex ideas.

\citeA{lewis1969convention} formalized the notion of language as a coordination problem by adopting work from game theory. Lewis defines a coordination problem as follows: \begin{quote} Two or more agents must each choose one of several alternative actions. Often all the agents have the same set of alternative actions, but that is not necessary. The outcomes the agents want to produce or prevent are determined jointly by the actions of all the agents (1969, pg. 8).
\end{quote} 
The key feature of these problems is that some combinations of the agents' choices are better than others: there are a set of equilibria of joint choices in which no agent would have a larger payoff had the agent alone changed their choice.

This broad framing can describe the dynamics of many social interactions. Take the above case of the barista and the customer, for example. We can model this interaction using a payoff matrix (Table 1). In the matrix, we represent the customer's payoff along the rows and the barista's payoff along the columns. There are two possible choices for  level of coffee in the cup---  full and  almost full --- and so each agent gets two rows or columns. 
\begin{table}[t]
\begin{center}
\begin{tabular}{l p{3cm} l p{3cm} l p{3cm} l}
 &  & \multicolumn{2}{c}{customer} \\ \cline{2-4} 
\multicolumn{1}{l|}{} & \multicolumn{1}{l|}{} & \multicolumn{1}{l|}{full} & \multicolumn{1}{l|}{almost full} \\ \cline{2-4} 
\multicolumn{1}{c|}{\multirow{2}{*}{barista}} & \multicolumn{1}{l|}{full} & \multicolumn{1}{l|}{0,0} & \multicolumn{1}{l|}{0,0} \\ \cline{2-4} 
\multicolumn{1}{c|}{} & \multicolumn{1}{l|}{almost full} & \multicolumn{1}{l|}{0,0} & \multicolumn{1}{l|}{1,1} \\ \cline{2-4} 
\end{tabular}
\caption{}
\end{center}
\end{table}
The agents relative payoffs are indicated in the cells, with the barista's on the left, and the customer's on the right. This happens to be a very simple equilibria --- there is one, and only one, possible set of actions in which is an equilibria. The customer prefers almost full and the barista fills the cup to almost full. The problem is that the barista does not know a priori where this equilibria lies, i.e. that the customer's pay off for almost full is 1, relative to 0 for full. To solve this coordination problem, the customer and barista make use of language.

More complicated coordination problems arise when there are multiple possible equilibria. Consider a weekend trip in which food and alcohol must be brought. To distribute the burden, half of the vacationers will bring food and the other half alcohol. In this case, the pay off matrix might look something like Table 2. Neither groups --- Group A or B --- have a strong preference about which of the two commodities they bring. However, what is important is that one group brings food and the other alcohol --- no one will be happy on a weekend trip with only food or only alcohol. There are thus two equilibiria, one each at a  set of choices where the two groups bring different things. By chance, the vacationers are equally likely to end up in a non-equilibrium as they are an equilibria. To ensure they end up in an equilibrium, they must coordinate, via language or some other means.

\begin{table}[t]
\begin{center}
\begin{tabular}{l p{3cm} l p{3cm} l p{3cm} l}
 &  & \multicolumn{2}{c}{Group A} \\ \cline{2-4} 
\multicolumn{1}{l|}{} & \multicolumn{1}{l|}{} & \multicolumn{1}{l|}{food} & \multicolumn{1}{l|}{alcohol} \\ \cline{2-4} 
\multicolumn{1}{c|}{\multirow{2}{*}{Group B}} & \multicolumn{1}{l|}{food} & \multicolumn{1}{l|}{0,0} & \multicolumn{1}{l|}{1,1} \\ \cline{2-4} 
\multicolumn{1}{c|}{} & \multicolumn{1}{l|}{alcohol} & \multicolumn{1}{l|}{1,1} & \multicolumn{1}{l|}{0,0} \\ \cline{2-4} 
\end{tabular}
\caption{}
\end{center}
\end{table}

What are the psychological forces that support these coordination games?
Zipf suggested a parsimonious way to think about the psychological sources at play in this interaction. He suggested that all equilibria could provided a parsimonious framework for thinking about the psychological forces that lead to the resolution of an equilibrium point in a coordination game. Zipf argued that dynamical systems were the result a single force.

other examples?  light example

\subsection{Language use as a coordination problem}

The application of this analysis to linguistic coordination problems is relatively straight forward. Consider the the utterance of a novel word.

As in the case of Zipf, 

Lay out Horn theory

Bayesian stuff, wilkes and gibes, scott

\subsection{Pragmatic equilibria reflected in the structure of language}
Consider four cases in which equililbtriam points predicted by Horn's theory of pragmatics are reflected in the structure of language: lexicon, semantics, semantics and words.

\subsubsection{Lexicon}
\subsubsection{Semantics}
\subsubsection{Words}
\subsubsection{Syntax}

%%%%%%%%% PART II %%%%%%%%% 
\section{Part II: The role of ontogenesis in the emergence of linguistic structure from pragmatic equilibria}


\bibliographystyle{apacite}
\bibliography{biblibrary}

\end{document}

